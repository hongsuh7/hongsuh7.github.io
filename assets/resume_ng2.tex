% LaTeX resume using res.cls
\documentclass{simplecv}

\usepackage[top = 0.5in, bottom = 0.5in, left = 0.5in, right = 0.5in]{geometry}
\usepackage{amsmath, amsthm, amssymb}
\usepackage{color, hyperref,comment,multicol}
\def\Z{{\mathbb{Z}}}

\begin{document}
\thispagestyle{empty}
\begin{center}
  {\Large\bf Hong Suh} \\
  \texttt{hong.suh7@gmail.com} ~~ $\diamond$ ~~ 408-807-1472 ~~ $\diamond$ ~~ San Francisco, CA \\
  %575 Los Palmos Dr. San Francisco, CA 94127\\
  \url{https://hongsuh7.github.io} ~~ $\diamond$ ~~ \url{https://github.com/hongsuh7} \\
\end{center}
\vspace{-0.5cm}

\section{EDUCATION} 

\noindent {\bf University of California, Berkeley}\hfill Berkeley, CA\\
{\it Ph.D, Mathematics} \hfill on leave \\
Specializations: Probability, Partial Differential Equations \hfill\\
Qualifying exams passed on April 9th, 2018 \\


\noindent {\bf University of California, Berkeley}\hfill Berkeley, CA\\
{\it M.A, Mathematics} (GPA: 3.85) \hfill May 2019  \\
%Advisor: Fraydoun Rezakhanlou\\
Graduate coursework: Probability, Numerical Differential Equations, Partial Differential Equations\\


\noindent {\bf Pomona College} \hfill Claremont, CA\\
{\sl B.A, Mathematics} (GPA: 3.89), cum laude, 3 department awards \hfill May 2016   \\
Coursework: Fundamentals of CS (functional programming), Data Structures \& Algorithms\\
%overall, 3.93 in Mathematics \\
                      % \sl will be bold italic in New Century Schoolbook (or
                % any postscript font) and just slanted in
          % Computer Modern (default) font
          %3 math department awards\\


\vspace{-0.6cm} 

%\begin{itemize}
%  \item[$\diamond$] NSF Graduate Research Fellowship Honorable Mention \hfill 2016
%  \item[$\diamond$] 3 Pomona College Mathematics Department Prizes \hfill 2014, 2015, 2016
%\end{itemize}
%\vspace{-0.4cm}

\rule{10cm}{0.5pt}

\vspace{-0.4cm}


\section{SKILLS}
\noindent {\it Programming:} Python, R, SQL, Java, Mathematica

\noindent {\it Tools/Packages:} PyTorch, scikit-learn, NumPy, CuPy, Pandas, tidyverse, Plotly

\noindent {\it Theory:} Deep Learning, Machine Learning, Probability, Numerical Differential Equations, Data Structures, Algorithms

\vspace{-0.2cm}

\rule{10cm}{0.5pt}

\vspace{-0.4cm}

\section{SELECT PROJECTS}

\begin{itemize}
  \item[$\diamond$] {\it Tennis win prediction model.} (Individual project) \hfill March -- August 2020 %(tech: Python, CuPy, plotly)
  \begin{itemize}
    \item Designed and implemented a prediction model for professional tennis player matchups using {\bf Pandas, NumPy}.
    \item Eliminated human supervision by automating hyperparameter selection using GPU optimization with {\bf CuPy}.
    \item Decreased log-loss error by about {\bf 1.5\%} compared to FiveThirtyEight's model.
    \item Created interactive graphics for users to experiment with predictions using {\bf Plotly}.
    %\item {\it Results.} Historically, hyperparameters have been hand-picked in Elo rating systems. Taking advantage of parallelization, I developed a tennis-specific model which automates hyperparameter selection. This model decreased log-loss error by about 1.5\% compared to FiveThirtyEight's model.
  \end{itemize}

  \item[$\diamond$] {\it Neural network model for image classification and generation.} (Individual project) \hfill June -- August 2020 %(tech: PyTorch, R)
  \begin{itemize}
    \item Developed a new normalizing flow architecture using masked convolutions and a modified {\bf neural ODE (NODE)} model for image generation using {\bf PyTorch} and Google Colab.%, achieving a bits-per-pixel metric of {\bf 100000} on the MNIST dataset.  %new high-performance neural network architecture for image classification and generation, modifying the methodology in recent normalizing flow papers.
    \item Conducted rigorous statistical tests on my modified NODE classifier model with {\bf adversarial training} to demonstrate its training speed-up and similar adversarial robustness compared to the vanilla NODE model.
    %\item Deployed application for users to interact with the deep generative image model.
    %\item Reduced computational cost for image generation using Neural ODEs by {\bf \textcolor{red}{x percent}} \textcolor{red}{while maintaining quality}.%Reduce computational cost of Neural ODEs, a neural network model, which are the legacy best in-class models for image generation. 
    %\item Created an interactive application to visualize the image generation process.%{\it Results.} Neural ODEs (NODEs) are the legacy best normalizing flow models for image generation, but they require high computational cost. I developed a new type of neural network inspired by NODEs that lowered computational cost by {\bf \textcolor{red}{x percent}} and can be used as a replacement for NODEs in image generation.
  \end{itemize}
\end{itemize}

\vspace{-0.4cm}

\rule{10cm}{0.5pt}

\vspace{-0.4cm}

\section{EXPERIENCE}
{\bf Math Teacher (grades 6-12)} \hfill June 2019 -- June 2020 \\
Proof School, San Francisco, CA
\begin{itemize}
  \itemsep0em 
  \item[$\diamond$] Created and executed daily 2-hour lesson plans for 8 to 16 students covering advanced math subjects---such as university-level linear algebra, number theory, and discrete probability---to kids who love math.
  \item[$\diamond$] Created and supervised mathematical programming projects involving fast matrix multiplication, singular value decomposition, pseudorandom number generators, and more.
\end{itemize}
{\bf Graduate Student Instructor (GSI) and Researcher} \hfill August 2016 -- May 2019 \\
UC Berkeley, Berkeley, CA
\begin{itemize}
  \itemsep0em
  \item[$\diamond$] Executed lectures and discussions as a GSI or primary lecturer to 20-50 undergraduate students in single-variable calculus, multivariable calculus, and linear algebra.
  \item[$\diamond$] Conducted research on stochastic interacting particle systems and presented at three seminars.
\end{itemize}

\vspace{-0.4cm}

\rule{10cm}{0.5pt}

\vspace{-0.4cm}


\section{PUBLICATIONS (AUTHORS IN ALPHABETICAL ORDER)}
\begin{itemize}
  \itemsep0em
  \item[$\diamond$] M. Asada, S. Manski, S. J. Miller, H. Suh, \emph{Fringe pairs in generalized MSTD sets}, Int. J. Number Theory 13.10 (2017): 2653-2675. %\textcolor{blue}{\url{http://arxiv.org/abs/1509.01657}}.
  %\item D. Burt, E. Goldstein, S. Manski, S. J. Miller, E. A. Palsson, H. Suh, \emph{Crescent configurations}, Integers \textbf{16} (2016), \#A38. %\textcolor{blue}{\url{http://arxiv.org/abs/1509.07220}}.
  \item[$\diamond$] P. Burkhardt, A. Z.-Y. Chan, G. Currier, S. R. Garcia, F. Luca, H. Suh, \emph{Visual Properties of Generalized Kloosterman sums}, J. Number Theory 160 (2016), 237-253. %\textcolor{blue}{\url{http://dx.doi.org/10.1016/j.jnt.2015.08.019}}.
  %\item P. Burkhardt, A. Z.-Y. Chan, G. Currier, S. R. Garcia, M. de Langis, B. Lutz, H. Suh, \emph{An exhibition of exponential sums: visualizing supercharacters}, Proceedings of Bridges 2015: Mathematics, Music, Art, Architecture, Culture, 475-478. %\textcolor{blue}{\url{http://archive.bridgesmathart.org/2015/bridges2015-475.pdf}}.
  %\item J. L. Brumbaugh, M. Bulkow, P. S. Fleming, L. A. Garcia German, S. R. Garcia, G. Karaali, M. Michal, A. P. Turner, H. Suh, \emph{Supercharacters, exponential sums, and the uncertainty principle}, J. Number Theory \textbf{144} (2014), 151-175. %\textcolor{blue}{\url{http://dx.doi.org/10.1016/j.jnt.2014.04.019}}.
  %\item A. Z.-Y. Chan, R. Domagalski, Y. H. Kim, S. K. Narayan, H. Suh, X. Zhang, \emph{Minimal scalings and structural properties of scalable frames}, preprint. %\textcolor{blue}{\url{http://arxiv.org/abs/1508.02266}}.
\end{itemize}


  %Neural ODEs (NODEs) are the legacy best-in-class flow models for image generation, but they require high computational cost. I developed a new type of neural network inspired by NODEs that lowered computational cost by \textcolor{red}{x percent} and can be used as a replacement for NODEs in image generation. This research is in progress.%Developed a new type of neural network which interpolates between residual networks and neural ODEs. Conducted statistical experiments to test the robustness of my network against various types of adversarial attacks compared to that of residual networks and neural ODEs. 

  %Historically, hyperparameters have been hand-picked in Elo rating systems. Taking advantage of parallelization, I developed a tennis-specific model which automates hyperparameter selection. This model decreased log-loss error by about 1.5\% compared to FiveThirtyEight's model. %%%automating hyperparameter selection. 
  % in two main ways: by incorporating surface and by eliminating the need to set hyperparameters by hand.
  % start with verbs
  %\begin{itemize}
  %  \item Current methods to take court surface into account are ad-hoc or complex. We simplified the integration of court surface into the model.
  %  \item In Elo rating systems, hyperparameters have historically been hand-picked for good reason: there is no effective method to optimize for hyperparameters. We demonstrated that in a random initial set of hyperparameters, a large proportion of them perform comparably to or better than current hand-picked hyperparameters. Taking advantage of parallelization, we were able to test a massive number of hyperparameters at once, which is a new and effective idea in Elo rating systems.
  %\end{itemize}
  % add results
  % objective, method, and results
  %\item[$\diamond$] {\it PhD research on stochastic homogenization for an exclusion process.}  
  %\begin{itemize}
  %  \item Established {\bf previously unresolved quantitative bounds} on the long-term statistics of a stochastic growth model, which is a class of models encompassing infection disease growth, forest fires, crystal growth, and more. 
    %\item Conducted {\bf Monte Carlo simulations} to study the statistic empirically.
    %\item \textcolor{red}{\it Results.} 
  %\end{itemize}
  %Infection disease growth, forest fires, and crystal growth can all be modeled with stochastic growth models. I established new quantitative bounds on the long-term statistics of a stochastic growth model. 
  %I studied long-term behavior of a stochastic growth model which can be used to model crystal growth, forest fires, infectious disease growth and more. I established new quantitative bounds on the model statistics. 
  %\item[$\diamond$] {\it Undergraduate research on fringe pairs in generalized MSTD sets.} 
  %\begin{itemize}
  %  \item Developed new algorithm to construct generalized MSTD sets, which are special integer sets, using {\bf Mathematica}. 
  %  \item Discovered the {\bf most ``extreme'' MSTD set} known at the time using the algorithm.
    %\item \textcolor{red}{\it Results.} 
  %  \item {\it Technologies.} Mathematica.
  %\end{itemize}
  %I led a project in a team of three to find new ways to construct generalized MSTD sets, which are special finite sets of integers, and found the most ``extreme'' MSTD set known at the time using our new methods. 
%\end{itemize}





%\section{PUBLICATIONS (AUTHORS IN ALPHABETICAL ORDER)}
%\begin{itemize}
%  \item M. Asada, S. Manski, S. J. Miller, H. Suh, \emph{Fringe pairs in generalized MSTD sets}, Int. J. Number Theory 13.10 (2017): 2653-2675. %\textcolor{blue}{\url{http://arxiv.org/abs/1509.01657}}.
  %\item D. Burt, E. Goldstein, S. Manski, S. J. Miller, E. A. Palsson, H. Suh, \emph{Crescent configurations}, Integers \textbf{16} (2016), \#A38. %\textcolor{blue}{\url{http://arxiv.org/abs/1509.07220}}.
%  \item P. Burkhardt, A. Z.-Y. Chan, G. Currier, S. R. Garcia, F. Luca, H. Suh, \emph{Visual Properties of Generalized Kloosterman sums}, J. Number Theory 160 (2016), 237-253. %\textcolor{blue}{\url{http://dx.doi.org/10.1016/j.jnt.2015.08.019}}.
  %\item P. Burkhardt, A. Z.-Y. Chan, G. Currier, S. R. Garcia, M. de Langis, B. Lutz, H. Suh, \emph{An exhibition of exponential sums: visualizing supercharacters}, Proceedings of Bridges 2015: Mathematics, Music, Art, Architecture, Culture, 475-478. %\textcolor{blue}{\url{http://archive.bridgesmathart.org/2015/bridges2015-475.pdf}}.
  %\item J. L. Brumbaugh, M. Bulkow, P. S. Fleming, L. A. Garcia German, S. R. Garcia, G. Karaali, M. Michal, A. P. Turner, H. Suh, \emph{Supercharacters, exponential sums, and the uncertainty principle}, J. Number Theory \textbf{144} (2014), 151-175. %\textcolor{blue}{\url{http://dx.doi.org/10.1016/j.jnt.2014.04.019}}.
  %\item A. Z.-Y. Chan, R. Domagalski, Y. H. Kim, S. K. Narayan, H. Suh, X. Zhang, \emph{Minimal scalings and structural properties of scalable frames}, preprint. %\textcolor{blue}{\url{http://arxiv.org/abs/1508.02266}}.
%\end{itemize}


% seems unnecessary
%{\sl Graduate Student} \hfill August 2016 -- May 2019\\
%Department of Mathematics, UC Berkeley
%\begin{itemize}
%  \item Studied probability and partial differential equations as a math Ph.D. student.
%\end{itemize}






%https://www.facebook.com/careers/jobs/593617728250395/
%https://instacart.careers/job/?id=2279948


\begin{comment}

\section{PRESENTATIONS}
\begin{itemize}
  \item ``One direction for nonconvex Aubry-Mather theory,'' Student Probability/PDE Seminar, UC Berkeley. November 2018.
 \item ``Stochastic Non-Homogenization of Non-Convex Hamilton-Jacobi Equations,'' Student Probability/PDE Seminar, UC Berkeley. October 2017.
 \item ``Fringe Pairs in Generalized MSTD Sets,'' Outstanding Poster in JMM Undergraduate Poster Session. Seattle, WA. January 2016.
 %\item ``Tight Frame Structure and Scalability,'' Joint Mathematics Meetings. San Antonio, TX. January 2015.
 %\item ``Frames in Finite Dimensions,'' SUMMR Conference. Allendale, MI. August 2014.
\end{itemize}



\section{EDUCATION EXPERIENCE}
{\sl Graduate Student Instructor} \hfill August 2016 -- Present \\
Department of Mathematics, UC Berkeley
\begin{itemize}
  \item Math 1A (Calculus 1) (Fall 2016, Fall 2017, Spring 2018)
  \item Math 1B (Calculus 2) (Spring 2017)
  \item Math 110 (Second course in linear algebra) (Fall 2018)
  \item Math 53 (Multivariable calculus) (Spring 2019)
\end{itemize}

{\sl Math 1A Lecturer} \hfill June 2017 -- August 2017 \\
Department of Mathematics, UC Berkeley


{\sl Teaching Assistant} \hfill January 2014 -- May 2016 \\
Mathematics Department, Pomona College
    \begin{itemize}
      \item Linear Algebra (Spring 2014, Spring 2015)
      \item Introduction to Analysis (Fall 2014)
      \item Real Analysis I (Fall 2015, Spring 2016)
      \item Complex Analysis (Fall 2015)
      \item Real Analysis II (Spring 2016)
    \end{itemize}

{\sl AARC Intern} \hfill August 2015 -- May 2016 \\
Asian American Resource Center, Pomona College, Claremont, CA
\begin{itemize}
  \item Planned and ran EmpowerU, a program that seeks to help local Filipino high school students critically engage with history, politics, and community. 
\end{itemize}

{\sl Math Tutor} \hfill August 2013 -- December 2013 \\
Upward Bound, Harvey Mudd College, Claremont, CA 
\begin{itemize}  \itemsep -2pt %reduce space between items
  \item Tutored calculus and statistics for local high school students once a week.
\end{itemize} 

{\sl Intern Math Teacher} \hfill June 2013 -- August 2013 \\
Breakthrough Collaborative, San Jose, CA 
\begin{itemize}  \itemsep -2pt %reduce space between items
  \item Taught Algebra II and Computer Science full-time to underserved $9^{\mbox{\scriptsize{th}}}$ grade students.
\end{itemize} 

{\sl Jumpstart Corps Member} \hfill September 2012 -- June 2013 \\
Jumpstart, Pitzer College, Claremont, CA
\begin{itemize}  \itemsep -2pt %reduce space between items
  \item Spent six hours per week in the classroom teaching underserved preschool children reading and writing.
\end{itemize}

\section{UNDER- \\GRADUATE RESEARCH}
{\sl Williams SMALL REU participant} \hfill June 2015 -- August 2015 \\
Department of Mathematics and Statistics, Williams College 



{\sl Research Assistant} \hfill November 2013 -- May 2015 \\
Mathematics Department, Pomona College



{\sl CMU REU participant} \hfill June 2014 -- August 2014 \\
Department of Mathematics, Central Michigan University

\section{REFERENCES}
{\sl Fraydoun Rezakhanlou (Advisor at UC Berkeley)} \\
Professor, Department of Mathematics, UC Berkeley\\
Email: \texttt{rezakhan@math.berkeley.edu}

{\sl Sug Woo Shin (Teaching reference at UC Berkeley)} \\
Associate Professor, Department of Mathematics, UC Berkeley\\
Email: \texttt{sug.woo.shin@berkeley.edu}

{\sl Janet Mota Navarro (Teaching mentor at Breakthrough Collaborative)} \\
Math teacher, Granite Oaks High School 

\end{comment}

\end{document}






